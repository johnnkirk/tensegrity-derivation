% file: oldderpl.tex
\documentstyle{article}
\topmargin -.35in\evensidemargin -0.5in\oddsidemargin -0.5in\headheight 0in\headsep 0in
% \date{20 June 1994}
\setlength{\textwidth}{7.25in}\setlength{\textheight}{10.35in}
\begin{document}
\begin{center} \Large\bf
Calculations and Graphs from J. Kirk's Original 6-Strut Tensegrity
\end{center}

Here is the verbatim text of the gnuplot program I wrote which generates
plots for the original formula for $d$ in terms of $\phi$ in John Kirk's
notes.

\begin{verbatim}
set terminal latex
set output "plot1.tex"
set xlabel "$\phi$ axis"
set ylabel "$d$ axis"
set nokey
set xtics -pi/4, pi/16
set ytics 0, 0.025
set size 1.0, 1.25
set nozeroaxis
set format x "$%.2f$"
set title "$d=\frac{4\sin^2\phi-1+\sqrt{1+8(1-2\sin^2\phi)^2}}{4\cos\phi}$"
d(x)=((4.*(sin(x)**2))-1.+sqrt(1.+8.*(1.-2.*(sin(x)*sin(x)))*(1.-2.*(sin(x)*sin(x)))))/(4.*cos(x))
plot [x=-pi/4:pi/4] [0.475:0.725] d(x) with lines
\end{verbatim}
Here is the graph of $d$ with respect to $\phi$:
	\begin{center}
		% GNUPLOT: LaTeX picture
\setlength{\unitlength}{0.240900pt}
\ifx\plotpoint\undefined\newsavebox{\plotpoint}\fi
\sbox{\plotpoint}{\rule[-0.200pt]{0.400pt}{0.400pt}}%
\begin{picture}(1500,1125)(0,0)
\font\gnuplot=cmr10 at 10pt
\gnuplot
\sbox{\plotpoint}{\rule[-0.200pt]{0.400pt}{0.400pt}}%
\put(220.0,113.0){\rule[-0.200pt]{4.818pt}{0.400pt}}
\put(198,113){\makebox(0,0)[r]{0.475}}
\put(1416.0,113.0){\rule[-0.200pt]{4.818pt}{0.400pt}}
\put(220.0,207.0){\rule[-0.200pt]{4.818pt}{0.400pt}}
\put(198,207){\makebox(0,0)[r]{0.5}}
\put(1416.0,207.0){\rule[-0.200pt]{4.818pt}{0.400pt}}
\put(220.0,302.0){\rule[-0.200pt]{4.818pt}{0.400pt}}
\put(198,302){\makebox(0,0)[r]{0.525}}
\put(1416.0,302.0){\rule[-0.200pt]{4.818pt}{0.400pt}}
\put(220.0,396.0){\rule[-0.200pt]{4.818pt}{0.400pt}}
\put(198,396){\makebox(0,0)[r]{0.55}}
\put(1416.0,396.0){\rule[-0.200pt]{4.818pt}{0.400pt}}
\put(220.0,491.0){\rule[-0.200pt]{4.818pt}{0.400pt}}
\put(198,491){\makebox(0,0)[r]{0.575}}
\put(1416.0,491.0){\rule[-0.200pt]{4.818pt}{0.400pt}}
\put(220.0,585.0){\rule[-0.200pt]{4.818pt}{0.400pt}}
\put(198,585){\makebox(0,0)[r]{0.6}}
\put(1416.0,585.0){\rule[-0.200pt]{4.818pt}{0.400pt}}
\put(220.0,679.0){\rule[-0.200pt]{4.818pt}{0.400pt}}
\put(198,679){\makebox(0,0)[r]{0.625}}
\put(1416.0,679.0){\rule[-0.200pt]{4.818pt}{0.400pt}}
\put(220.0,774.0){\rule[-0.200pt]{4.818pt}{0.400pt}}
\put(198,774){\makebox(0,0)[r]{0.65}}
\put(1416.0,774.0){\rule[-0.200pt]{4.818pt}{0.400pt}}
\put(220.0,868.0){\rule[-0.200pt]{4.818pt}{0.400pt}}
\put(198,868){\makebox(0,0)[r]{0.675}}
\put(1416.0,868.0){\rule[-0.200pt]{4.818pt}{0.400pt}}
\put(220.0,963.0){\rule[-0.200pt]{4.818pt}{0.400pt}}
\put(198,963){\makebox(0,0)[r]{0.7}}
\put(1416.0,963.0){\rule[-0.200pt]{4.818pt}{0.400pt}}
\put(220.0,1057.0){\rule[-0.200pt]{4.818pt}{0.400pt}}
\put(198,1057){\makebox(0,0)[r]{0.725}}
\put(1416.0,1057.0){\rule[-0.200pt]{4.818pt}{0.400pt}}
\put(220.0,113.0){\rule[-0.200pt]{0.400pt}{4.818pt}}
\put(220,68){\makebox(0,0){$-0.79$}}
\put(220.0,1037.0){\rule[-0.200pt]{0.400pt}{4.818pt}}
\put(372.0,113.0){\rule[-0.200pt]{0.400pt}{4.818pt}}
\put(372,68){\makebox(0,0){$-0.59$}}
\put(372.0,1037.0){\rule[-0.200pt]{0.400pt}{4.818pt}}
\put(524.0,113.0){\rule[-0.200pt]{0.400pt}{4.818pt}}
\put(524,68){\makebox(0,0){$-0.39$}}
\put(524.0,1037.0){\rule[-0.200pt]{0.400pt}{4.818pt}}
\put(676.0,113.0){\rule[-0.200pt]{0.400pt}{4.818pt}}
\put(676,68){\makebox(0,0){$-0.20$}}
\put(676.0,1037.0){\rule[-0.200pt]{0.400pt}{4.818pt}}
\put(828.0,113.0){\rule[-0.200pt]{0.400pt}{4.818pt}}
\put(828,68){\makebox(0,0){$0.00$}}
\put(828.0,1037.0){\rule[-0.200pt]{0.400pt}{4.818pt}}
\put(980.0,113.0){\rule[-0.200pt]{0.400pt}{4.818pt}}
\put(980,68){\makebox(0,0){$0.20$}}
\put(980.0,1037.0){\rule[-0.200pt]{0.400pt}{4.818pt}}
\put(1132.0,113.0){\rule[-0.200pt]{0.400pt}{4.818pt}}
\put(1132,68){\makebox(0,0){$0.39$}}
\put(1132.0,1037.0){\rule[-0.200pt]{0.400pt}{4.818pt}}
\put(1284.0,113.0){\rule[-0.200pt]{0.400pt}{4.818pt}}
\put(1284,68){\makebox(0,0){$0.59$}}
\put(1284.0,1037.0){\rule[-0.200pt]{0.400pt}{4.818pt}}
\put(1436.0,113.0){\rule[-0.200pt]{0.400pt}{4.818pt}}
\put(1436,68){\makebox(0,0){$0.79$}}
\put(1436.0,1037.0){\rule[-0.200pt]{0.400pt}{4.818pt}}
\put(220.0,113.0){\rule[-0.200pt]{292.934pt}{0.400pt}}
\put(1436.0,113.0){\rule[-0.200pt]{0.400pt}{227.410pt}}
\put(220.0,1057.0){\rule[-0.200pt]{292.934pt}{0.400pt}}
\put(45,585){\makebox(0,0){$d$ axis}}
\put(828,23){\makebox(0,0){$\phi$ axis}}
\put(828,1102){\makebox(0,0){$d=\frac{4\sin^2\phi-1+\sqrt{1+8(1-2\sin^2\phi)^2}}{4\cos\phi}$ }}
\put(220.0,113.0){\rule[-0.200pt]{0.400pt}{227.410pt}}
\put(220,989){\usebox{\plotpoint}}
\multiput(220.58,972.12)(0.492,-5.106){21}{\rule{0.119pt}{4.067pt}}
\multiput(219.17,980.56)(12.000,-110.559){2}{\rule{0.400pt}{2.033pt}}
\multiput(232.58,856.17)(0.493,-4.144){23}{\rule{0.119pt}{3.331pt}}
\multiput(231.17,863.09)(13.000,-98.087){2}{\rule{0.400pt}{1.665pt}}
\multiput(245.58,751.99)(0.492,-3.900){21}{\rule{0.119pt}{3.133pt}}
\multiput(244.17,758.50)(12.000,-84.497){2}{\rule{0.400pt}{1.567pt}}
\multiput(257.58,662.65)(0.492,-3.383){21}{\rule{0.119pt}{2.733pt}}
\multiput(256.17,668.33)(12.000,-73.327){2}{\rule{0.400pt}{1.367pt}}
\multiput(269.58,585.18)(0.492,-2.909){21}{\rule{0.119pt}{2.367pt}}
\multiput(268.17,590.09)(12.000,-63.088){2}{\rule{0.400pt}{1.183pt}}
\multiput(281.58,519.18)(0.493,-2.281){23}{\rule{0.119pt}{1.885pt}}
\multiput(280.17,523.09)(13.000,-54.088){2}{\rule{0.400pt}{0.942pt}}
\multiput(294.58,461.67)(0.492,-2.133){21}{\rule{0.119pt}{1.767pt}}
\multiput(293.17,465.33)(12.000,-46.333){2}{\rule{0.400pt}{0.883pt}}
\multiput(306.58,412.77)(0.492,-1.789){21}{\rule{0.119pt}{1.500pt}}
\multiput(305.17,415.89)(12.000,-38.887){2}{\rule{0.400pt}{0.750pt}}
\multiput(318.58,372.11)(0.493,-1.369){23}{\rule{0.119pt}{1.177pt}}
\multiput(317.17,374.56)(13.000,-32.557){2}{\rule{0.400pt}{0.588pt}}
\multiput(331.58,337.43)(0.492,-1.272){21}{\rule{0.119pt}{1.100pt}}
\multiput(330.17,339.72)(12.000,-27.717){2}{\rule{0.400pt}{0.550pt}}
\multiput(343.58,308.13)(0.492,-1.056){21}{\rule{0.119pt}{0.933pt}}
\multiput(342.17,310.06)(12.000,-23.063){2}{\rule{0.400pt}{0.467pt}}
\multiput(355.58,283.68)(0.492,-0.884){21}{\rule{0.119pt}{0.800pt}}
\multiput(354.17,285.34)(12.000,-19.340){2}{\rule{0.400pt}{0.400pt}}
\multiput(367.58,263.29)(0.493,-0.695){23}{\rule{0.119pt}{0.654pt}}
\multiput(366.17,264.64)(13.000,-16.643){2}{\rule{0.400pt}{0.327pt}}
\multiput(380.58,245.65)(0.492,-0.582){21}{\rule{0.119pt}{0.567pt}}
\multiput(379.17,246.82)(12.000,-12.824){2}{\rule{0.400pt}{0.283pt}}
\multiput(392.00,232.92)(0.496,-0.492){21}{\rule{0.500pt}{0.119pt}}
\multiput(392.00,233.17)(10.962,-12.000){2}{\rule{0.250pt}{0.400pt}}
\multiput(404.00,220.92)(0.652,-0.491){17}{\rule{0.620pt}{0.118pt}}
\multiput(404.00,221.17)(11.713,-10.000){2}{\rule{0.310pt}{0.400pt}}
\multiput(417.00,210.93)(0.758,-0.488){13}{\rule{0.700pt}{0.117pt}}
\multiput(417.00,211.17)(10.547,-8.000){2}{\rule{0.350pt}{0.400pt}}
\multiput(429.00,202.93)(0.874,-0.485){11}{\rule{0.786pt}{0.117pt}}
\multiput(429.00,203.17)(10.369,-7.000){2}{\rule{0.393pt}{0.400pt}}
\multiput(441.00,195.93)(1.267,-0.477){7}{\rule{1.060pt}{0.115pt}}
\multiput(441.00,196.17)(9.800,-5.000){2}{\rule{0.530pt}{0.400pt}}
\multiput(453.00,190.94)(1.797,-0.468){5}{\rule{1.400pt}{0.113pt}}
\multiput(453.00,191.17)(10.094,-4.000){2}{\rule{0.700pt}{0.400pt}}
\multiput(466.00,186.95)(2.472,-0.447){3}{\rule{1.700pt}{0.108pt}}
\multiput(466.00,187.17)(8.472,-3.000){2}{\rule{0.850pt}{0.400pt}}
\put(478,183.17){\rule{2.500pt}{0.400pt}}
\multiput(478.00,184.17)(6.811,-2.000){2}{\rule{1.250pt}{0.400pt}}
\put(490,181.67){\rule{3.132pt}{0.400pt}}
\multiput(490.00,182.17)(6.500,-1.000){2}{\rule{1.566pt}{0.400pt}}
\put(503,180.67){\rule{2.891pt}{0.400pt}}
\multiput(503.00,181.17)(6.000,-1.000){2}{\rule{1.445pt}{0.400pt}}
\put(539,180.67){\rule{3.132pt}{0.400pt}}
\multiput(539.00,180.17)(6.500,1.000){2}{\rule{1.566pt}{0.400pt}}
\put(552,181.67){\rule{2.891pt}{0.400pt}}
\multiput(552.00,181.17)(6.000,1.000){2}{\rule{1.445pt}{0.400pt}}
\put(564,182.67){\rule{2.891pt}{0.400pt}}
\multiput(564.00,182.17)(6.000,1.000){2}{\rule{1.445pt}{0.400pt}}
\put(576,183.67){\rule{2.891pt}{0.400pt}}
\multiput(576.00,183.17)(6.000,1.000){2}{\rule{1.445pt}{0.400pt}}
\put(588,184.67){\rule{3.132pt}{0.400pt}}
\multiput(588.00,184.17)(6.500,1.000){2}{\rule{1.566pt}{0.400pt}}
\put(601,186.17){\rule{2.500pt}{0.400pt}}
\multiput(601.00,185.17)(6.811,2.000){2}{\rule{1.250pt}{0.400pt}}
\put(613,188.17){\rule{2.500pt}{0.400pt}}
\multiput(613.00,187.17)(6.811,2.000){2}{\rule{1.250pt}{0.400pt}}
\put(625,189.67){\rule{3.132pt}{0.400pt}}
\multiput(625.00,189.17)(6.500,1.000){2}{\rule{1.566pt}{0.400pt}}
\put(638,191.17){\rule{2.500pt}{0.400pt}}
\multiput(638.00,190.17)(6.811,2.000){2}{\rule{1.250pt}{0.400pt}}
\put(650,193.17){\rule{2.500pt}{0.400pt}}
\multiput(650.00,192.17)(6.811,2.000){2}{\rule{1.250pt}{0.400pt}}
\put(662,194.67){\rule{2.891pt}{0.400pt}}
\multiput(662.00,194.17)(6.000,1.000){2}{\rule{1.445pt}{0.400pt}}
\put(674,196.17){\rule{2.700pt}{0.400pt}}
\multiput(674.00,195.17)(7.396,2.000){2}{\rule{1.350pt}{0.400pt}}
\put(687,197.67){\rule{2.891pt}{0.400pt}}
\multiput(687.00,197.17)(6.000,1.000){2}{\rule{1.445pt}{0.400pt}}
\put(699,199.17){\rule{2.500pt}{0.400pt}}
\multiput(699.00,198.17)(6.811,2.000){2}{\rule{1.250pt}{0.400pt}}
\put(711,200.67){\rule{3.132pt}{0.400pt}}
\multiput(711.00,200.17)(6.500,1.000){2}{\rule{1.566pt}{0.400pt}}
\put(724,201.67){\rule{2.891pt}{0.400pt}}
\multiput(724.00,201.17)(6.000,1.000){2}{\rule{1.445pt}{0.400pt}}
\put(736,202.67){\rule{2.891pt}{0.400pt}}
\multiput(736.00,202.17)(6.000,1.000){2}{\rule{1.445pt}{0.400pt}}
\put(748,203.67){\rule{2.891pt}{0.400pt}}
\multiput(748.00,203.17)(6.000,1.000){2}{\rule{1.445pt}{0.400pt}}
\put(760,204.67){\rule{3.132pt}{0.400pt}}
\multiput(760.00,204.17)(6.500,1.000){2}{\rule{1.566pt}{0.400pt}}
\put(515.0,181.0){\rule[-0.200pt]{5.782pt}{0.400pt}}
\put(785,205.67){\rule{2.891pt}{0.400pt}}
\multiput(785.00,205.17)(6.000,1.000){2}{\rule{1.445pt}{0.400pt}}
\put(773.0,206.0){\rule[-0.200pt]{2.891pt}{0.400pt}}
\put(859,205.67){\rule{2.891pt}{0.400pt}}
\multiput(859.00,206.17)(6.000,-1.000){2}{\rule{1.445pt}{0.400pt}}
\put(797.0,207.0){\rule[-0.200pt]{14.936pt}{0.400pt}}
\put(883,204.67){\rule{3.132pt}{0.400pt}}
\multiput(883.00,205.17)(6.500,-1.000){2}{\rule{1.566pt}{0.400pt}}
\put(896,203.67){\rule{2.891pt}{0.400pt}}
\multiput(896.00,204.17)(6.000,-1.000){2}{\rule{1.445pt}{0.400pt}}
\put(908,202.67){\rule{2.891pt}{0.400pt}}
\multiput(908.00,203.17)(6.000,-1.000){2}{\rule{1.445pt}{0.400pt}}
\put(920,201.67){\rule{2.891pt}{0.400pt}}
\multiput(920.00,202.17)(6.000,-1.000){2}{\rule{1.445pt}{0.400pt}}
\put(932,200.67){\rule{3.132pt}{0.400pt}}
\multiput(932.00,201.17)(6.500,-1.000){2}{\rule{1.566pt}{0.400pt}}
\put(945,199.17){\rule{2.500pt}{0.400pt}}
\multiput(945.00,200.17)(6.811,-2.000){2}{\rule{1.250pt}{0.400pt}}
\put(957,197.67){\rule{2.891pt}{0.400pt}}
\multiput(957.00,198.17)(6.000,-1.000){2}{\rule{1.445pt}{0.400pt}}
\put(969,196.17){\rule{2.700pt}{0.400pt}}
\multiput(969.00,197.17)(7.396,-2.000){2}{\rule{1.350pt}{0.400pt}}
\put(982,194.67){\rule{2.891pt}{0.400pt}}
\multiput(982.00,195.17)(6.000,-1.000){2}{\rule{1.445pt}{0.400pt}}
\put(994,193.17){\rule{2.500pt}{0.400pt}}
\multiput(994.00,194.17)(6.811,-2.000){2}{\rule{1.250pt}{0.400pt}}
\put(1006,191.17){\rule{2.500pt}{0.400pt}}
\multiput(1006.00,192.17)(6.811,-2.000){2}{\rule{1.250pt}{0.400pt}}
\put(1018,189.67){\rule{3.132pt}{0.400pt}}
\multiput(1018.00,190.17)(6.500,-1.000){2}{\rule{1.566pt}{0.400pt}}
\put(1031,188.17){\rule{2.500pt}{0.400pt}}
\multiput(1031.00,189.17)(6.811,-2.000){2}{\rule{1.250pt}{0.400pt}}
\put(1043,186.17){\rule{2.500pt}{0.400pt}}
\multiput(1043.00,187.17)(6.811,-2.000){2}{\rule{1.250pt}{0.400pt}}
\put(1055,184.67){\rule{3.132pt}{0.400pt}}
\multiput(1055.00,185.17)(6.500,-1.000){2}{\rule{1.566pt}{0.400pt}}
\put(1068,183.67){\rule{2.891pt}{0.400pt}}
\multiput(1068.00,184.17)(6.000,-1.000){2}{\rule{1.445pt}{0.400pt}}
\put(1080,182.67){\rule{2.891pt}{0.400pt}}
\multiput(1080.00,183.17)(6.000,-1.000){2}{\rule{1.445pt}{0.400pt}}
\put(1092,181.67){\rule{2.891pt}{0.400pt}}
\multiput(1092.00,182.17)(6.000,-1.000){2}{\rule{1.445pt}{0.400pt}}
\put(1104,180.67){\rule{3.132pt}{0.400pt}}
\multiput(1104.00,181.17)(6.500,-1.000){2}{\rule{1.566pt}{0.400pt}}
\put(871.0,206.0){\rule[-0.200pt]{2.891pt}{0.400pt}}
\put(1141,180.67){\rule{2.891pt}{0.400pt}}
\multiput(1141.00,180.17)(6.000,1.000){2}{\rule{1.445pt}{0.400pt}}
\put(1153,181.67){\rule{3.132pt}{0.400pt}}
\multiput(1153.00,181.17)(6.500,1.000){2}{\rule{1.566pt}{0.400pt}}
\put(1166,183.17){\rule{2.500pt}{0.400pt}}
\multiput(1166.00,182.17)(6.811,2.000){2}{\rule{1.250pt}{0.400pt}}
\multiput(1178.00,185.61)(2.472,0.447){3}{\rule{1.700pt}{0.108pt}}
\multiput(1178.00,184.17)(8.472,3.000){2}{\rule{0.850pt}{0.400pt}}
\multiput(1190.00,188.60)(1.797,0.468){5}{\rule{1.400pt}{0.113pt}}
\multiput(1190.00,187.17)(10.094,4.000){2}{\rule{0.700pt}{0.400pt}}
\multiput(1203.00,192.59)(1.267,0.477){7}{\rule{1.060pt}{0.115pt}}
\multiput(1203.00,191.17)(9.800,5.000){2}{\rule{0.530pt}{0.400pt}}
\multiput(1215.00,197.59)(0.874,0.485){11}{\rule{0.786pt}{0.117pt}}
\multiput(1215.00,196.17)(10.369,7.000){2}{\rule{0.393pt}{0.400pt}}
\multiput(1227.00,204.59)(0.758,0.488){13}{\rule{0.700pt}{0.117pt}}
\multiput(1227.00,203.17)(10.547,8.000){2}{\rule{0.350pt}{0.400pt}}
\multiput(1239.00,212.58)(0.652,0.491){17}{\rule{0.620pt}{0.118pt}}
\multiput(1239.00,211.17)(11.713,10.000){2}{\rule{0.310pt}{0.400pt}}
\multiput(1252.00,222.58)(0.496,0.492){21}{\rule{0.500pt}{0.119pt}}
\multiput(1252.00,221.17)(10.962,12.000){2}{\rule{0.250pt}{0.400pt}}
\multiput(1264.58,234.00)(0.492,0.582){21}{\rule{0.119pt}{0.567pt}}
\multiput(1263.17,234.00)(12.000,12.824){2}{\rule{0.400pt}{0.283pt}}
\multiput(1276.58,248.00)(0.493,0.695){23}{\rule{0.119pt}{0.654pt}}
\multiput(1275.17,248.00)(13.000,16.643){2}{\rule{0.400pt}{0.327pt}}
\multiput(1289.58,266.00)(0.492,0.884){21}{\rule{0.119pt}{0.800pt}}
\multiput(1288.17,266.00)(12.000,19.340){2}{\rule{0.400pt}{0.400pt}}
\multiput(1301.58,287.00)(0.492,1.056){21}{\rule{0.119pt}{0.933pt}}
\multiput(1300.17,287.00)(12.000,23.063){2}{\rule{0.400pt}{0.467pt}}
\multiput(1313.58,312.00)(0.492,1.272){21}{\rule{0.119pt}{1.100pt}}
\multiput(1312.17,312.00)(12.000,27.717){2}{\rule{0.400pt}{0.550pt}}
\multiput(1325.58,342.00)(0.493,1.369){23}{\rule{0.119pt}{1.177pt}}
\multiput(1324.17,342.00)(13.000,32.557){2}{\rule{0.400pt}{0.588pt}}
\multiput(1338.58,377.00)(0.492,1.789){21}{\rule{0.119pt}{1.500pt}}
\multiput(1337.17,377.00)(12.000,38.887){2}{\rule{0.400pt}{0.750pt}}
\multiput(1350.58,419.00)(0.492,2.133){21}{\rule{0.119pt}{1.767pt}}
\multiput(1349.17,419.00)(12.000,46.333){2}{\rule{0.400pt}{0.883pt}}
\multiput(1362.58,469.00)(0.493,2.281){23}{\rule{0.119pt}{1.885pt}}
\multiput(1361.17,469.00)(13.000,54.088){2}{\rule{0.400pt}{0.942pt}}
\multiput(1375.58,527.00)(0.492,2.909){21}{\rule{0.119pt}{2.367pt}}
\multiput(1374.17,527.00)(12.000,63.088){2}{\rule{0.400pt}{1.183pt}}
\multiput(1387.58,595.00)(0.492,3.383){21}{\rule{0.119pt}{2.733pt}}
\multiput(1386.17,595.00)(12.000,73.327){2}{\rule{0.400pt}{1.367pt}}
\multiput(1399.58,674.00)(0.492,3.900){21}{\rule{0.119pt}{3.133pt}}
\multiput(1398.17,674.00)(12.000,84.497){2}{\rule{0.400pt}{1.567pt}}
\multiput(1411.58,765.00)(0.493,4.144){23}{\rule{0.119pt}{3.331pt}}
\multiput(1410.17,765.00)(13.000,98.087){2}{\rule{0.400pt}{1.665pt}}
\multiput(1424.58,870.00)(0.492,5.106){21}{\rule{0.119pt}{4.067pt}}
\multiput(1423.17,870.00)(12.000,110.559){2}{\rule{0.400pt}{2.033pt}}
\put(1117.0,181.0){\rule[-0.200pt]{5.782pt}{0.400pt}}
\end{picture}

	\end{center}

Here is Calc's processing of the $d(x)$ used in the gnuplot program:
\begin{verbatim}
((4.*(sin(x)**2))-1.+sqrt(1.+8.*(1.-2.*(sin(x)*sin(x)))*(1.-2.*(sin(x)*sin(x)))))/(4.*cos(x))
\end{verbatim}

Which Calc simplifies to:
\begin{verbatim}
(4. sin(x)^2 - 1. + sqrt(1. + 8. (1. - 2. sin(x)^2)^2)) / 4. cos(x)
\end{verbatim}
\newpage
In ``big'' notation this becomes:
\begin{verbatim}
		    ____________________________
	 2         |                        2 2
4. sin(x)  - 1. + \| 1. + 8. (1. - 2. sin(x) )
------------------------------------------------
		   4. cos(x)
\end{verbatim}

Now to get the derivative.  Here is the formula for $d(x)$ as it was
when GNU Calc began it's derivation of the derivative.
\begin{verbatim}
(4. sin(x)^2 - 1. + sqrt(1. + 8. (1. - 2. sin(x)^2)^2)) / 4. cos(x)

% [calc-mode: language: nil]
% [calc-mode: symbolic: nil]
% [calc-mode: fractions: nil]
% [calc-mode: angles: rad]
\end{verbatim}

Here is the derivative $d'(x)$:
\begin{verbatim}
(8. sin(x) cos(x) 
   - 32. (1. - 2. sin(x)^2) sin(x) cos(x) / sqrt(1. + 8. (1. - 2. sin(x)^2)^2)) 
  / 4. cos(x) 
  + 0.25 (4. sin(x)^2 - 1. + sqrt(1. + 8. (1. - 2. sin(x)^2)^2)) sin(x) 
      / cos(x)^2
\end{verbatim}

Which simplifies by collecting the $cos(x)$ terms:
\begin{verbatim}
2. sin(x) - 8. sin(x) (1. - 2. sin(x)^2) / sqrt(8. (1. - 2.  sin(x)^2)^2 + 1.)
+ 0.25 sin(x) (4. sin(x)^2 + sqrt(8. (1. - 2. sin(x)^2)^2 + 1.) - 1.) / cos(x)^2
\end{verbatim}
In "big" notation:
\begin{verbatim}
% [calc-mode: language: big]
									 ____________________________
				      2                            2    |                   2 2
	     8. sin(x) (1. - 2. sin(x) )     0.25 sin(x) (4. sin(x)  + \| 8. (1. - 2. sin(x) )  + 1.  - 1.)
2. sin(x) - ------------------------------ + --------------------------------------------------------------
	      ____________________________                                    2
	     |                   2 2                                    cos(x)
	    \| 8. (1. - 2. sin(x) )  + 1.
\end{verbatim}
\newpage
Here is the gnuplot program to plot this:
\begin{verbatim}
set output "plot2.tex"
set autoscale
set xlabel "$\phi$ axis"
set nokey
set ytics 0, 0.025
set size 1.25, 1.5
set nozeroaxis
set xzeroaxis
set format x "$%.2f$"
set ylabel "$d'$ axis"
set xtics -pi/4, pi/16
set ytics -1.75, 0.25
set title "$d' = \frac{2 \cos\phi \sin\phi -
\frac{8\cos\phi \sin\phi ( 1-2\sin^2\phi ) }{\sqrt{8(1-2\sin^2\phi)^2 + 1}}}{\cos\phi} +
\frac{(4\sin^2\phi + \sqrt{8(1-2\sin^2\phi)^2} -1)}{4\cos^2\phi} $\\[0.5cm]"
f(x) = 2.*sin(x) -
8.*sin(x)*(1. - 2.*(sin(x)*sin(x))) / sqrt(8.*(1. - 2.*(sin(x)*sin(x)))*(1. - 2.*(sin(x)*sin(x))) + 1.)
+ (0.25*sin(x))*(4.*(sin(x)*sin(x)) + sqrt(8.*(1. - 2.*(sin(x)*sin(x)))*(1. - 2.*(sin(x)*sin(x))) + 1.) - 1.)
/ (cos(x)*cos(x))
plot [x=-pi/4:pi/4] [-1.75:1.75] f(x) with lines
\end{verbatim}
Here is the plot of the derivative:
\begin{center}
	\input plot2.tex
\end{center}
\end{document}
